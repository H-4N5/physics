%maxwell fisheye lens, luneburg lens; indukcyjność rachunek; rotating charged sphere

\documentclass[a4paper, twocolumn]{article}

\usepackage{amssymb}

\usepackage[utf8]{inputenc}
\usepackage[T1]{fontenc}
\usepackage[polish]{babel}
\selectlanguage{polish}

\usepackage{amsmath}
\usepackage{enumerate}
\usepackage{graphicx}
\usepackage{hyperref}
\usepackage{mathrsfs}
\usepackage{physics}
\usepackage{lipsum}
\usepackage{subfiles}

%For n-th order differential use \dd[n]{x}
%For n-th order derivative use \dv[n]{f}{x}
%For n-th order partial derivative use \pdv[n]{f}{x}

\newcommand{\e}{\mathrm{e}}
\newcommand{\im}{\mathrm{i}}

\usepackage{pgfplots}
\pgfplotsset{compat=1.15}
\usepackage{mathrsfs}
\usetikzlibrary{arrows}
\usepackage{circuitikz}
\usepackage{tikz}
\usetikzlibrary{shapes.geometric}

\begin{document}
\subsubsection*{Zadanie 1.} 
Dwa psy ciągną sanie. W chwili pokazanej na rysunku prędkości psów są skierowane wzdłuż lin i mają wartości \(v_1\) i \(v_2\). Liny tworzą ze sobą kąt \(\alpha\). Oblicz wartość prędkości \(V\) sanek.
\begin{figure}[h!]
    \centering
    \includegraphics[scale=0.2]{Bez tytułu.png}
    \label{fig:my_label}
\end{figure}
\subsubsection*{Zadanie 2.} 
Prostopadłościenny, jednorodny blok skalny o masie \(M\) o podstawie kwadratu o boku \(b\) i wysokości \(h>b\) spoczywa na ziemi. Krawędzie bloku są lekko zaokrąglone. O krawędź bloku (na środku tej krawędzi) oparto lekką drabinę, której drugi koniec spoczywa na ziemi w odległości \(b\) od bloku. Na drabinę zaczął powoli wchodzić człowiek o masie \(m\) (wymiary człowieka można przyjąć za bardzo niewielkie). W trakcie wspinaczki nie występowało tarcie między drabiną a blokiem, lecz drabina się nie poruszała. Blok także był nieruchomy, jednak w pewnym momencie w trakcie wspinaczki człowieka blok zaczął się przewracać (obracać wokół jednej z krawędzi bez przesuwania po ziemi). Wyznacz warunek, który muszą spełniać parametry \(M\), \(m\), \(b\) i \(h\) oraz minimalny współczynnik tarcia \(\mu\) bloku o ziemię, dla którego opisana sytuacja jest możliwa.

\subsubsection*{Zadanie 3.}
Jeden koniec jednorodnego pręta o długości \(l\) jest przyczepiony do sufitu w punksie \(A\) za pomocą lekkiej nici również o długości \(l\). Drugi koniec pręta (\(B\)) spoczywa na gładkiej podłodze. W chwili początkowej odcinek \(AB\) jest prostopadły do sufitu i \(|AB|=h\), przy czym zachodzi \(l<h<2l\). Następnie koniec pręta zaczyna ślizgać się po podłodze w taki sposób, że nić jest stale napięta. Wyznacz maksymalną szybkość środka masy pręta podczas dalszego ruchu.

\subsubsection*{Zadanie 4.}
A light rod with length \(l\) is connected to the horizontal surface with a hinge; a small sphere of mass \(m\) is connected to the end of the rod. Initially the rod is vertical and the sphere rests against the block of mass \(M\). The system is left to freely move and after a certain time the block loses contact with the surface of the block — at the moment when the rod forms an angle \(\phi=\pi/6\) with the horizontal. Find the ratio of masses \(M/m\) and the velocity \(V\) of the block at the moment of separation.

\subsubsection*{Zadanie 5.} 
Gaz jednoatomowy jest ogrzewany w taki sposób, że jego ciepło molowe wynosi \(2R\). Podczas ogrzewania objętość gazu podwoiła się. Jak zmieniła się temperatura gazu?

\subsubsection*{Zadanie 6.}
Wyznacz stosunek pojemności bardzo cienkiego, przewodzącego dysku o promieniu \(R\) do pojemności przewodzącej kuli o takim samym promieniu.

\subsubsection*{Zadanie 7.} 
Rozstrzygnij, czy pojemność cienkościennej, przewodzącej półsfery o średnicy \(f\) jest większa niż \(\pi\epsilon_0f\).

\begin{figure}[ht]
    \centering
    \includegraphics[scale=0.24]{3.png}
\end{figure}

\subsubsection*{Zadanie 8.}
Jeden koniec sztywnego, nieważkiego pręta o długości \(l\) jest przymocowany do punktu \((0,0,0)\), zaś do drugiego końca przymocowano niewielką kulkę o masie \(m\) i ładunku \(q\). Cały układ znajduje się w nieważkości w jednorodnym polu magnetycznym \(\mathbf{B}=[0,0,B]\) skierowanym pionowo do góry. Początkowo kulka znajduje się w punkcie \((l,0,0)\) i ma prędkość \(\mathbf{v}_0=[0,0,v]\) skierowaną pionowo do góry. Wyznacz maksymalną współrzędną \(z\) osiąganą przez kulkę w trakcie jej ruchu.
\newpage

\subsubsection*{Zadanie 9.}

Dwie równoległe, ustawiono pionowo metalowe, kwadratowe płyty o boku \(a\), są utrzymywane nad powierzchnią nieprzewodzącej cieczy o gęstości \(\rho\), tak że ich dolne krawędzie dotykają powierzchni cieczy. Płytki są w odległości \(d\) od siebie. Po podłączeniu płytek do akumulatora, który utrzymuje stałe napięcie \(U\), ciecz unosi się między płytami, ledwo sięgając ich górnych krawędzi. Oblicz stałą dielektryczną cieczy. Pomiń wpływ napięcia powierzchniowego.

\subsubsection*{Zadanie 10.}

Pocisk wystrzelony z ziemi eksplodował na trzy fragmenty o równej masie w najwyższym punkcie trajektorii. Jeden z fragmentów spadł po czasie \(\tau\) (licząc od chwili wybuchu); pozostałe dwa fragmenty upadły jednocześnie po czasie \(2\tau\) od chwili wybuchu. Oblicz wysokość, na jakiej eksplodował pocisk.

\subsubsection*{Zadanie 11.}
Statek kosmiczny w kształcie stożka wykorzystuje ciśnienie promieniowania słonecznego do oddalania się od Słońca. Oś stożka wskazuje bezpośrednio na Słońce. Stożkowa powierzchnia statku jest pomalowana na czarno. Astronauci próbują następnie zwiększyć swoje przyspieszenie, pokrywając powierzchnię stożkową materiałem silnie odbijającym światło. Ku ich przerażeniu przyspieszenie zmniejsza się o 30\%. Oblicz kąt rozwarcia stożka.

\subsubsection*{Zadanie 12.}
7 niewielkich identycznych kul (każda o masie \(m\)) o numerach 1,2,...,7 zostało połączonych w łańcuch (połączono pary (1,2), (2,3),..., (6,7)) lekkimi pętami (o długości \(l\) każdy). Połączenia kul z prętami były ruchome (tj. pręty połączone z jedną kulą mogły się obracać wokół kuli niezależnie od siebie). Następnie unieruchomiono skrajne kule w odległości \(L\) od siebie (na tej samej wysokości), pozwalając reszcie kul zwisać swobodnie. Okazało się, że pręt łączący kule o numerach 3 i 4 był nachylony pod kątem \(30^\circ\) do poziomu. Wyraź wartość \(L\) za pomocą \(l\).

\subsubsection*{Zadanie 13.}
Dany jest metalowy pręt, którego przekrój poprzeczny jest kołem o promieniu \(r\). Pręt ten wygięto w obręcz o promieniu \(R\gg r\), przy czym końce pręta były od siebie odległe o \(d\ll r\), zaś powierzchnie końców pręta były płaskie i równoległe do siebie. Metal, z którego wykonano pręt, ma rezystywność \(\varrho\), zaś obręcz ma masę \(m\). Obręcz tę umieszczono w obszarze, w którym składowa \(z\) pola magnetycznego wynosi \(\alpha z\) dla pewnej stałej \(\alpha\) i pole to wykazuje symetrię obrotową wokół osi \(z\). Obręcz była prostopadła do osi \(z\), przy czym oś \(z\) byłą osią symetrii obręczy. W chwili \(t=0\) przez obręcz nie płynął prąd, zaś współrzędna \(z\) jej środka wynosiła \(z_0\). Przyspieszenie grawitacyjne było stałe, skierowane przeciwnie do osi \(z\) i równe co do wartości \(g\). Obręcz spadała swobodnie, nie obracała się, ani nie doformowała. Okazało się, że przyspieszenie obręczy zależało od czasu zgodnie z następującym wzorem
\begin{equation*}
    a(t)=A+Be^{Ct}\,.
\end{equation*}
Wyznacz stałe \(A\), \(B\), \(C\).

\subsubsection*{Zadanie 14.}
Na rysunku wszystkie trzy woltomierze są idealne i jednakowe. Każdy opornik ma taki sam opór \(R\). Napięcie na baterii wynosi \(U\). Oblicz wskazania poszczególnych woltomierzy.
\begin{figure}[ht]
     \begin{center}
      \begin{circuitikz}[american voltages, scale=0.7]
      \draw
      (0,0) to [battery] (6,0)
      to [short] (6,2)
      to [generic,*-*] (4,2)
      to [generic,*-*] (2,2)
      to [generic, *-*] (0,2)
      to [short] (0,0)
      (2,4) to [voltmeter, l_=$V_2$] (2,2)
      (2,4) to [voltmeter, l=$V_3$,*-] (4,4)
      to [short] (4,2) 
      (6,2) to [short] (6,6)
      to [generic] (4,6)
      to [generic] (2,6)
      to [generic, *-] (0,6)
      to [short] (0,4)
      (2,6) to [voltmeter, l_=$V_1$] (2,4)
      (0,4) to [short] (0,2);
      \end{circuitikz}
  \end{center}
     \label{fig:circuit1}
 \end{figure}

\subsubsection*{Zadanie 15.}
Wyznacz moment bezwładności jednorodnego prostopadłościanu o masie \(M\) i krawędziach długości \(a\), \(b\), \(c\) względem osi obrotu zawierającej najdłuższą przekątną prostopadłościanu.

\subsubsection*{Zadanie 16.}
Układ, którego schemat przedstawiono poniżej, zawiera idealne ogniwo i dwa rezystory. Woltomierz użyty do pomiaru napięć na rezystorach i na baterii daje następujące wskazania: 2 V, 3 V, 6 V. Jakie są rzeczywiste napięcia na rezystorach w tym układzie?
\begin{figure}[ht]
     \begin{center}
      \begin{circuitikz}[american voltages, scale=0.8]
      \draw
      (0,0) to [battery] (6,0)
      to [short] (6,2)
      to [generic, l_=$R_2$] (3,2)
      to [generic, l_=$R_1$] (0,2)
      to [short] (0,0);
      \end{circuitikz}
  \end{center}
     \label{fig:circuit2}
 \end{figure}

\subsubsection*{Zadanie 17.}
Na poniższym rysunku każdy prostokąt oznacza rezystor o oporze \(1\,\Omega\). Wyznacz opór zastępczy między punktami \(A\) i \(B\).

\begin{figure}[ht]
    \centering
    \includegraphics[scale=0.3]{crct3.png}
    \label{fig:circuit3}
\end{figure}

\subsubsection*{Zadanie 18.}

\begin{figure}[ht]
    \centering
    \includegraphics[scale=0.28]{herring.png}
    \label{fig:Hering}
\end{figure}

Consider the system depicted in the picture below. The galvanometer remains fixed in the lab, while the magnet is moved towards the galvanometer and the tips of the leads slide around it. Determine total charge \(Q\) which passed through the galvanometer if the areas of magnet and loop are \(A_1\), \(A_2\) and the resistance of the galvanometer is \(R\).

\subsubsection*{Zadanie 19.}
Rozważmy dwie kule o promieniach \(a\), \(b\), które przecinają się pod kątem prostym (tj. płaszczyzny styczne do sfer w dowolnym punkcie ich przecięcia są prostopadłe). Wykonano metalowy odlew zewnętrznej powierzchni takiej bryły. W punkcie \(X\) w odległości \(c\) od środka kuli o promieniu \(a\) i odległości \(d\) od środka kuli o promieniu \(b\) umieszczono ładunek punktowy o wartości \(q\). Wyznacz całkowity ładunek wyindukowany na przewodniku, jeżeli metalowy odlew został uziemiony.

\subsubsection*{Zadanie 20.}
Dana jest przezroczysta kula o promieniu \(R\). Współczynnik załamania światła zależy od odległości \(r\) od środka kuli według wzoru
\begin{equation*}
    n(r)=\frac{R+a}{r+a}\,,
\end{equation*}
gdzie \(a>0\) jest stałą. Na kulę pod kątem \(\alpha\) pada promień światła. Wyznacz najmniejszą odległość tego promienia od środka kuli.

\subsubsection*{Zadanie 21.}
Dana jest cienka powłoka sferyczna o promieniu \(R\). Gęstość powierzchniowa masy powłoki w dowolnym punkcie \(P\) jest dana zależnością
\begin{equation*}
    \sigma(P)=\frac{\lambda}{SP^3}\,,
\end{equation*}
gdzie \(\lambda\) jest stałą, a \(S\) jest pewnym ustalonym punktem wewnątrz powłoki sferycznej (możesz uznać, że znane są jego współrzędne). Wiedząc, że odległość między punktem \(S\) i środkiem powłoki \(C\) wynosi \(f\) wyznacz siłę działającą na jednostkową masę punktową umieszczoną w punkcie \(X\) na zewnątrz powłoki, takim, że \(SX=CX=d\).

\subsubsection*{Zadanie 22.}
In 1939, Cullwick reported an experiment, sketched above/below, in which a steady current \(I\) flowed in a fixed wire along the axis of an annular metal cylinder that moved with velocity \(v\) parallel to its axis. A galvanometer was connected to contacts on the inner and outer radii of the cylinder, which contacts remain fixed in the lab as the cylinder slid past them. Cullwick found that reading of the galvanometer \underline{did not} depend on the permeability of the cylinder, which could be made of copper or of iron. Explain theoretically the results of the described experiment and calculate the measured EMF assuming that the cylinder had inner radius \(a\) and outer radius \(b\).

\begin{figure}
    \centering
    \includegraphics[scale=0.3]{cullwick.png}
    \label{fig:cullwick}
\end{figure}

\subsubsection*{Zadanie 23.}
W dwóch czarnych skrzynkach z zaznaczonymi wyprowadzeniami znajdują się narysowane obwody elektryczne. Podaj warunki jakie muszą spełniać parametry \(R_1\), \(R_2\), \(R_3\), \(R_4\), \(C\), \(C'\), aby niemożliwe było rozróżnienie obu układów za pomocą jakichkolwiek zewnętrznych pomiarów elektrycznych.

\begin{figure}[ht]
    \centering
    \begin{circuitikz}[american voltages, scale=0.8]
    \draw (0,0) to [generic, l_=$R_1$, -*] (0,-2)
    to [short] (1,-2)
    to [generic, l=$R_2$] (1,-4)
    to [short, -*] (0,-4);
    \draw (0,-2) to [short] (-1,-2)
    to [C, l_=$C$] (-1,-4)
    to [short] (0,-4)
    to [short] (0,-5) {};
    \draw (0,0) to [short, -o] (-1,0);
    \draw (0,-5) to [short,-o] (-1,-5);
    
    \draw (4,-1) to [short] (6,-1)
    to [generic,l=$R_4$] (6,-5);
    \draw (4,-1) to [short] (4,-1)
    to [generic, l_=$R_3$] (4,-3)
    to [C, l_=$C'$] (4,-5)
    to [short] (6,-5);
    \draw (4,-1) to [short,*-o] (3,-1);
    \draw (4,-5) to [short,*-o] (3,-5);
    
    \end{circuitikz}
    \label{fig:circuit4}
\end{figure}

\subsubsection*{Zadanie 24.}
Przewrócony ciężki stożkowy lejek postawiono na równej poziomej płaszczyźnie pokrytej arkuszem gumy. Węższy otwór lejka zakończony jest cienką rurką, przez którą można do wnętrza lejka nalewać wodę. Okazuje się, że woda zaczyna wyciekać spod lejka, gdy wysokość jej poziomu w rurce (mierząc od podłoża) wynosi \(h\). Oblicz masę lejka \(M\), jeśli pole przekroju jego szerszego otworu spoczywającego na podłożu wynosi \(S\), a wysokość lejka (bez rurki) wynosi \(H\).

\subsubsection*{Zadanie 25.}
Udowodnij, że moment bezwładności jednorodnego sześcianu o masie \(M\) i krawędzi długości \(a\) względem dowolnej osi przechodzącej przez środek masy sześcianu wynosi \(\frac{1}{6}Ma^2\).

\subsubsection*{Zadanie 26.}
Z materiału o rezystywności \(\varrho\) wykonano rezystor w kształcie stożka ściętego o wysokości \(h\) i promieniach podstaw \(a\), \(b\) (\(a<b\)). Do podstaw stożka przyłożono elektrody będące kołami o promieniach odpowiednio \(a\) i \(b\). Wyznacz przybliżoną wartość rezystancji takiego układu. Podaj warunki, jakie muszą spełniać parametry \(a\), \(b\), \(h\), aby wyznaczona wartość była bliska wartości rzeczywistej.

\subsubsection*{Zadanie 27.}
Gładki, długi pręt tworzy kąt \(\alpha\) z podłożem. Niewielki pierścień o masie \(m\) jest nanizany na pręt i może swobodnie się po nim ślizgać. Do pierścienia przymocowana jest lekka nitka na końcu której znajduje się niewielka kulka o masie \(M\). Początkowo pierścień jest unieruchomiony, a nić wisi pionowo. Następnie pierścień zostaje puszczony. Wyznacz przyspieszenie kulki tuż po zwolnieniu pierścienia.

\subsubsection*{Zadanie 28.}
A long cylinder of radius \(R\) has uniform magnetization \(\mathbf{M}\) transverse to its axis. Find the magnetic field \(\mathbf{B}\) inside the cylinder.

\subsubsection*{Zadanie 29.}
Na płytkę szklaną grubości \(100,25\lambda\) pada prostopadle promień światła laserowego, którego długość fali wewnątrz płytki wynosi \(\lambda\). Gdy światło o natężeniu \(I\) pada na powierzchnię styku powietrze--szkło lub szkło--powietrze, wiązka przechodząca będzie miała natężenie \(rI\) (dla pewnego ustalonego \(0<r<1\)), zaś odbita \((1-r)I\). Wiązka odbita od granicy faz szkło--powietrze od strony powietrza zmienia fazę o \(\pi\), zaś wiązka przechodzące (w obie strony) lub odbite od granicy faz od strony szkła nie zmieniają fazy. Wyznacz natężenie wiązki odbitej od płytki. Natężenie światła jest proporcjonalne do kwadratu amplitudy natężenia pola elektrycznego w wiązce. Przyjmij, że szkło nie pochłania światła.

\subsubsection*{Zadanie 30.}
W pojemniku wyposażonym w tłok znajduje się \(n\) moli dwuatomowego gazu doskonałego o temperaturze \(T\). Pojemność cieplna pojemnika wynosi \(C\). Gaz jest w doskonałym kontakcie termicznym z pojemnikiem -- wymienia z nim ciepło bardzo szybko. Pojemnik jest izolowany termicznie od otoczenia (nie wymienia energii cieplnej z otoczeniem w żaden sposób). Początkowo objętość gazu w pojemniku wynosiła \(V\). Wyznacz pracę potrzebną do przesunięcia tłoka tak, by objętość gazu zmniejszyła się do \(V/2\).

\subsubsection*{Zadanie 31.}
Ciężar \(p\) jest zawieszony na trzech stalowych drutach wykonanych z tego samego materiału, przyczepionych do sufitu w punktach odpowiednio \(A\), \(B\), \(C\). Punkt \(D\) jest miejscem połączenia wszystkich trzech drutów. Wyznacz siły, jakimi rozciągane są poszczególne druty. Zaniedbaj masy drutów. Druty spełniają prawo Hooke'a z bardzo dużym modułem Younga -- tzn. takim, że wprawdzie rozciągnięcie każdego z drutów jest niezerowe, ale jest ono zaniedbywalnie małe w porównaniu z odległością między dowolnymi dwoma danymi punktami.

\subsubsection*{Zadanie 32.}
Sides of the heptagon are resistors of resistance \(1\,\Omega\) and the diagonals are resistors of resistance \(2\,\Omega\). Find the full resistance between two adjacent points as a rational number in \(\Omega\).

\subsubsection*{Zadanie 33.}
Na płaszczyźnie zaznaczono 25 punktów postaci \((x,y)\), gdzie \(x,y\in\{1,2,...,5\}\). Każde dwa punkty odległe o \(1\) połączono rezystorem o wartości \(1\,\Omega\) otrzymując kratę \(4\times4\) oporników. Wyznacz opór zastępczy tego układu (jako wymierną liczbę wyrażoną w \(\Omega\)) między punktami \((1,1)\) i \((5,5)\).

\subsubsection*{Zadanie 34.}
W dużym jednorodnym ośrodku o współczynniku przewodzenia ciepła \(\kappa\) znajduje się cienki dysk o promieniu \(a\) i bardzo dużej przewodności cieplnej. W płaszczyźnie dysku umieszczono współśrodkową z nim, okrągłą pętlę o promieniu \(b=a+\epsilon\), gdzie \(0<\epsilon\ll a\), wykonaną z drutu oporowego, którą następnie podłączono do źródła napięcia. Moc cieplna wydzielana na elemencie wynosiła \(P\). Wyznacz temperaturę dysku \(T_0\) w stanie stacjonarnym, tj. stanie, w którym temperatura każdego punktu nie zależy od czasu. Przyjmij. że temperatura w punkcie bardzo odległym od dysku wynosi \(T_\infty\). Pomiń wymianę ciepła na skutek promieniowania.
\medskip

\textbf{Przewodzenie ciepła}\\
Niech \(T(\mathbf{r})\) będzie polem skalarnym opisującym rozkład temperatury w przestrzeni w stanie ustalonym. Oznaczmy \(\mathbf{G}=-\nabla T\), wówczas jeśli przez \(\Phi\) oznaczymy strumień \(\mathbf{G}\) przez pewną powierzchnię zamkniętą to w jednorodnym ośrodku o przewodności cieplnej \(\kappa\) zachodzi
\begin{equation*}
    \Phi=\frac{P}{\kappa}\,,
\end{equation*}
gdzie \(P\) jest całkowitym ciepłem przepływającym przez tą powierzchnię zamkniętą w jednostce czasu.
\medskip

\textbf{Wzory, które mogą być przydatne}
\begin{equation*}
    \int \frac{1}{\sqrt{1-x^2}}\dd{x}=\arcsin(x)+\text{const.}
\end{equation*}
\end{document}
